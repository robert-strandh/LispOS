\chapter{Device drivers}

\section{Introduction}

The purpose of a device driver is to act as an intermediate layer
between an operating-specific API that is common for a group of
similar devices and vendor-specific interfaces for individual types of
devices. 

An important part of writing device drivers for a Lisp operating
system is therefore to specify the different groups of devices and the
corresponding operating-specific API for each group.%
\footnote{As everything else in this document, this chapter is open to
  discussion.  More so here, because I have no prior experience in
  defining device-driver APIs. -- RS}

\section{Disk drivers}

\Defprotoclass{disk}

This is the root class of all disk device classes.  

\Defclass{standard-disk}

\Defgeneric {write-block} {disk block address \key (hang t)}

Write a block of data to the disk.  The parameter \textit{disk} is an
instance of the class \textit{disk}, and \textit{block} is a a vector
of type \texttt{(simple-array (unsigned-byte 8) (*))}.  The size of
\textit{block} must be a power of 2, and the address must be aligned
to the size of \textit{block}.  If the size of \textit{block} is
\emph{smaller} than the native block size of \textit{disk} then a
request to read an entire native block will be issued and the result
will be stored in a temporary location.  Part of the native block in
the temporary location will then be overwritten by the contents of
\textit{block}.  Finally, the contents of the temporary location will
be written to the device.  If the size of \textit{block} is
\emph{greater} than the native block size of \textit{disk}, then
several native blocks will be written from \textit{block}.

If \textit{hang} is \emph{true} (the default), then the current
process will be suspended until the transfer is complete.
If \textit{hang} is \emph{false} the call will return a \emph{time
  stamp}, which is a fixnum that can later be used in an invocation of
the \texttt{wait} function.

\Defgeneric {read-block} {disk block address \key (hang t)}

Write a block of data to the disk.  The parameter \textit{disk} is an
instance of the class \textit{disk}, and \textit{block} is a a vector
of type \texttt{(simple-array (unsigned-byte 8) (*))}.  The size of
\textit{block} must be a power of 2, and the address must be aligned
to the size of \textit{block}.  If the size of \textit{block} is
\emph{smaller} than the native block size of \textit{disk} then a
request to read an entire native block will be issued and the result
will be stored in a temporary location.  Then a part of that native
block will be copied to \textit{block}.  If the size of \textit{block}
is \emph{greater} than the native block size of \textit{disk}, then
several native blocks will be read into \textit{block}.

If \textit{hang} is \emph{true} (the default), then the current
process will be suspended until the transfer is complete.
If \textit{hang} is \emph{false} the call will return a \emph{time
  stamp}, which is a fixnum that can later be used in an invocation of
the \texttt{wait} function.

\Defgeneric {wait} {disk time-stamp \key (hang t)}

Wait for a transfer to complete.  If the transfer is already complete,
then this function returns immediately.  If not, then the current
process is suspended until the transfer is complete. 

If \textit{hang} is \emph{true} (the default), then the current
process will be suspended until the transfer is complete.
If \textit{hang} is \emph{false}, then \texttt{wait} will return
\emph{true} if the transfer is complete, and \emph{false} otherwise. 
