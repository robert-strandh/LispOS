\chapter{Address space}

The system will have a single address space, as opposed to a separate
address space for each process.  We are targeting 64-bit
architectures, which allows us to consider all disk memory as being
part of the address space.  Basically, main memory is a cache for the
disk(s), and the virtual address of an object refers to its location
on disk (with some exceptions as discussed below).

Half the address space (say the upper half) contains thread-local
storage.  Each thread has a local heap and a stack.  For example, the
thread-local address space can be 16 or 32MiB of which the local heap
can be around 4MiB (smaller than the size of the cache).  Thread-local
storage must have associated physical disk memory. 

The other half of the address space (say the lower half) contains
mainly the global heap, aside from a small amount of space dedicated
to interrupt handlers and other architecture-imposed data structures.
This part of the address space is divided in (roughly two) halves.
Only half of it requires associated physical disk memory.  Aside from
the architecture-imposed data structures. The two halves basically map
to the same disk space.  The reason for this organization is that it
makes it easy to implement the global garbage collector
\seechap{chap-garbage-collection}.  The method invented by Kermany et
al \cite{Kermany:2006:CCI:1133981.1134023} divides the heap address
space in two halves, but requires physical memory only for a little
more than half of it. 

