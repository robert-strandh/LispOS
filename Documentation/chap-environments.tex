\chapter{Environments}
\label{chap-environments}

Recall that an \emph{environment} is a mapping from \emph{names} to
\emph{objects}.  This mapping consists of a set of \emph{bindings}. 

When a user is created in the system, a \emph{default global
  environment} is created for that user.   The global environment of a
user consists of a \emph{system-wide} environment and a
\emph{user-specific} environment.

The system-wide environment consists of bindings that are themselves
immutable (i.e., the user is not allowed to alter the binding) such as
the binding of the symbol \texttt{cl:length} to the function that
returns the length of a sequence.  The objects of these bindings are
also immutable, such as the length function itself.  The system-wide
environment is the same for every user, allowing the installation of
software that is immediately visible to all users. 

The user-specific environment consists of bindings that are created by
the user.  These bindings are of three different kinds:

\begin{itemize}
\item Bindings created by the user for instance as a result of
  executing a \texttt{defparameter} or \texttt{defun} form.
\item Default system-wide bindings that can be altered by the user,
  such as the value of \texttt{*print-base*}.
\item Immutable bindings where the \emph{object} can be modified by
  the user, such as system-defined generic functions to which the user
  is allowed to add specific methods.  Each user has a private copy of
  such objects.
\end{itemize}

When a function or method object is created as a result of calling
\texttt{compile} on a lambda expression, or as a result of loading a
\emph{fasl} file, the object is \emph{linked} to the current global
environment, in that external references are then resolved.  When such
a function or method object is given to a different user, that
different user can execute it, but external references in it will
still refer to the environment into which it was compiled or loaded.

This mechanism provides an efficient method of protection.  A user A
can grant controlled access to part of his or her global environment
by allowing a user B to execute a function made available to him or
her through the \emph{object store}.  \seechap{chap-object-store} In a
traditional modern operating system such as \unix{}, this kind of
controlled access required the use of the \emph{setuid} mechanism,
simply because in such a system there is no way to access an object
other than through the global file system, and the accessing user must
have the right permissions to access the object.

The same mechanism can be used by the system itself to protect objects
that would be unwise to give users direct access to, such as disks or
printers. 

